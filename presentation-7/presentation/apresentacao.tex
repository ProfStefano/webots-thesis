\documentclass{beamer}

\mode<presentation>{
\usetheme{Dresden}
\setbeamercovered{transparent}
\usecolortheme{lsc}
}

\mode<handout>{
  % tema simples para ser impresso
  \usepackage[bar]{beamerthemetree}
  % Colocando um fundo cinza quando for gerar transparências para serem impressas
  % mais de uma transparência por página
  \beamertemplatesolidbackgroundcolor{black!5}
}

\usepackage{amsmath,amssymb}
\usepackage[brazil]{varioref}
\usepackage[english,brazil]{babel}
\usepackage[utf8]{inputenc}
%\usepackage[latin1]{inputenc}
\usepackage{graphicx}
\usepackage{listings}
\usepackage{url}
\usepackage{colortbl}
\usepackage[ruled, linesnumbered]{algorithm2e}
\usepackage{amsmath}
\usepackage{hyperref}

\newcommand\Fontvi{\fontsize{6}{10}\selectfont}

\beamertemplatetransparentcovereddynamic

\addtobeamertemplate{navigation symbols}{}{%
    \usebeamerfont{footline}%
    \usebeamercolor[fg]{footline}%
    \hspace{1em}%
    \insertframenumber/\inserttotalframenumber
}

\setbeamercolor{footline}{fg=black}
\setbeamerfont{footline}{series=\bfseries}

\title[Machine Learning for Probabilistic Robotics with Webots]{Machine Learning for Probabilistic Robotics with Webots}
\author[Joan Gerard]{%
  Joan Gerard\inst{1} \\
  Promotor: Prof. Gianluca Bontempi \inst{1}}
  \institute[ULB]{
  \inst{1}%
     Universit\'e Libre de Bruxelles}

% Se comentar a linha abaixo, irá aparecer a data quando foi compilada a apresentação  
\date{April 1th, 2020}

\AtBeginSection[]{
  \begin{frame}<beamer>
    \frametitle{Table of Contents}
    \tableofcontents[currentsection,currentsubsection]
  \end{frame}
}

\begin{document}

\begin{frame}
\titlepage
\end{frame}

\begin{frame}
\frametitle{Table of Contents}
\tableofcontents
\end{frame}

\section{What is new?}
\frame{
	\frametitle{Collection of data}
	\begin{itemize}
		\item Running simulation for 30 minutes in fast mode: 1757372 collected samples.
		\item Sensor noise decreased to 0.05 for each sensor.
		\item Samples taken continuously at each robot time step.
	\end{itemize}
}

\frame{
	\frametitle{Neural Network Architecture}
	
	\begin{itemize}
		\item Optimizer: rmsprop
		\item Input: 3
		\item Sequential: 16; relu.
		\item Sequential: 32; relu.
		\item Sequential: 64; relu.
		\item Sequential: 16; relu.
		\item Output: 1
	\end{itemize}
}

\pgfdeclareimage[height=3cm]{MAE}{figs/mae.png}
\pgfdeclareimage[height=3cm]{LOSS}{figs/loss.png}

\frame{
	\frametitle{Model training}
	\begin{itemize}
		\item Avoid data normalization.
		\item Retrain the 8 NN models using mini-batch mode instead of stochastic mode using batch size = 64 for 50 epochs.
		\item Assess the model using k-fold for only one sensor: sensor 3.
		\item The sensors in the front of the robot (1, 8) perform worst than the rest: loss: 0.0887 - mae: 0.0832
		\item $NMSE = 0.023$
	\end{itemize}
	\centering
	\pgfuseimage{MAE}
	\pgfuseimage{LOSS}
}


\pgfdeclareimage[height=3cm]{EXPERIMENT}{figs/particles-distance-error.png}
\pgfdeclareimage[height=3cm]{ARENA}{figs/arena.png}
\section{Experiment}
\frame {
	\begin{itemize}
		\item Resampling at $t$ mod $2 == 0$. Where $t$ is the robot time step.
		\item Run the simulation with parameters: $\sigma_{xy} = 0.001$, $\sigma_{\theta} = 2$
		\item Run the simulation for $t = 1000$ with $K = 30, 100, 500, 2000$ where $K$ is the number of particles.
		\item Use \textbf{online learning}: refit the models at each time step for 50 epochs with new data: robot estimated state $\rightarrow$ sensor measurements. 
	\end{itemize}
	
	\begin{columns}
		\column{6cm}
			\pgfuseimage{EXPERIMENT}
			\pgfuseimage{ARENA}
		\column{3cm}
		\centering
		{ \tiny
			\textbf{RMSE (cm)}
			\begin{itemize}
				\item Odometry $15.53$
				\item 30 particles w. refit 12.90
				\item 30 particles $ 8.69$
				\item 100 particles $ 6.73$
				\item 500 particles $4.93$
				\item 2000 particles $4.36$
			\end{itemize}}
	\end{columns}
}


\end{document}






